\documentclass[dvipdfmx]{jsarticle}
\usepackage[hiresbb]{graphicx}
\usepackage[dvipdfmx]{hyperref}
\usepackage{pxjahyper}
\usepackage[top=10mm, bottom=20mm, left=15mm, right=15mm]{geometry}

\hypersetup{
colorlinks=true,
linkcolor=black,
urlcolor=blue
}

\author{e.wakabayashi-aa}
\title{PowerShellについて}



\begin{document}


\maketitle

\begin{abstract}
DOSのバッチコマンドについて調べていたときに知ったのがPowerShellです。
コマンドプロンプト(cmd.exe)の後継らしきものであることは分かったものの
コマンドは冗長でオプションは意味不明。情報は少なくとくに初心者向けのものはほとんど見つかりません。この状況はいまもあまり変わっていません。
PowerShellは管理者向けに作られたものであり、.NET(ドットネット)は開発者の使うものであり、そもそも初心者が使うことを想定していないのです。
5000円もした分厚い解説書は半年間埃をかぶるままでしたが必要にせまられ少しづつ解明していきました。
データを構造化できるpscustomobjectと外部からエクセルVBAを操作する仕組みinteropを知ってから手放せないもになりました。ほとんどすべてのことがPowerShellでできます。
すべての情報はネット上にあります。しかしその勘所は暗黙知です。
\end{abstract}

\tableofcontents

\newpage

\section{PowerShellとは}

マイクロソフトが開発したコマンド・ライン・インターフェース(CLI)シェル、およびスクリプト言語。
一言でいえばDOSプロンプト(DOS窓)の後継ですが機能はそれをはるかに上回ります。
バッチを走らせることができるだけでなく開発者から見たWindowsともいえる \textbf{.NET Framework}を使えるのでおよそWindowsできることはすべてできます。
\textbf{Microsoft.Office.Interop.Excel}を介せば外部からエクセル・マクロ(VB)と同等のコマンドでエクセル・ファイルをセル単位で操作することが可能です。


\section{どこにある}
Windows標準搭載(Windows7~)なのでダウンロードしたりインストールしたりする必要はない。インストール済みなのでスタート・メニューを開けば出てくるのだがPCによってその場所は微妙に異なる。ちなみに現在私が使用中のPC(Windows10)ではそのもずばり「Windows PowerShell」フォルダとして「W」項目内で見つけることがえきる。またフォルダ・メニューの「ファイル」に「Windows PowerShell を開く」項目がありそこから実行することができる。本体(実行ファイル)は\verb+C:\WINDOWS\system32\WindowsPowerShell\v1.0+にある。アプリケーションとしては変わった場所だがDOS(cmd.exe)の後継と考えればそれも納得がいく。

\subsection{ISEか否か}
ただの\textbf{Windows PowerShell}と\textbf{Windows PowerShell ISE}の2種類がある。
ISEとは「統合開発環境」のこと。ISE無しはDOS窓の後継としての意味しかない(ように思える)のでもっぱら使うのはISEのほうを、拡張子「.ps1」の関連づけもISEのほうにしています。

\subsection{32bitか64bitか}
末尾に\textbf{(x86)}の付いたのが32bit、付いてないのが64bit。基本的にどちらを使っても同じですが64bitマシンなので64bitを使っています。
この違いが問題になるのはデータ・ベースへの接続を行うときです。
ODBCドライバには32bit用しかないものがあります。そのときはPowerShellも32bit用でなければなりません。
32bitか64bitかは\textbf{[System.Environment]::Is64BitProcess}で確認できます。


\section{実行権限}
標準搭載ですがPCによってはスクリプトが実行できないよう権限が設定されていることがあります。
スクリプトはDOSのバッチ・ファイルのようにコマンドの記述されたテキスト・ファイルです。拡張子はDOSでは\textbf{.bat}でしたがPowerShellでは\textbf{.ps1}です。
現在の実行権限は\textbf{Get-ExecutionPolicy}で確認し、
\textbf{Set-ExecutionPolicy}で設定します。
実行権限にはいくつか種類があり私のPCでは\textbf{RemoteSigned}で、スクリプトの実行ができます。
ただしインターネットや共有サーバーからダウンロードしたスクリプトにはデジタル署名が必要となります。

\subsection{例}
現在のユーザーに対して権限を設定し、結果を確認する。(全ユーザに対して設定するには管理者権限が必要)

\begin{quote}
\begin{verbatim}
PS> Set-ExecutionPolicy -ExecutionPolicy RemoteSigned -Scope CurrentUser

PS> 
PS> 
PS> Get-ExecutionPolicy
RemoteSigned
\end{verbatim}
\end{quote}




\section{DOSコマンドとの比較}
DOSでできることはすべてPoserShellでできる。DOSに比べてPoserShellのコマンドは長いがタブ補間が効くので見た目ほどタイピングは大変ではない。

\hspace{3mm}

\begin{tabular}{ll}
DOS & PowerShell \\
\hline
DIR & Get-ChildItem \\
DEL & Remove-Item \\
CD & Set-Location \\
ECHO & Write-Output \\
DATE & Get-Date \\
& Set-Date \\
FOR & foreach \\
& ForEach-Object
\end{tabular}

\subsection{タブ補間}
コマンドの途中でTABキーを押すと候補が表示される。頭2、3文字打てばたいていのコマンドは入力することができる。これはコマンド本体だけでなくオプションにもその設定値に対しても作動する。

\subsection{エイリアスの使用}
DOSに慣れたユーザーのために、またキー入力を少なくするために、別名(エイリアス)を持つコマンドがある。dirやcdくらいなら問題ないが、\textbf{\%}(For-EachObject)や\textbf{?}(Where-Object)となるとある程度PowerShellを使い慣れた人でないと想像がつかず特殊文字であるために検索も難しい。他人と未来の自分のためにエイリアスは使わないことを勧める。一番時間かかるのはPCの性能不足や最適化されていないプログラムコードではなく人間が悩んで考えている時間であるから。同様の理由でオプションの省略も好ましくない。


\section{プロンプト}
プロンプトとはコマンドラインの左端に常に表示される文字列です。デフォルトではこれがカレント(現在)ディレクトリのフルパスになっており、深いフォルダで作業するときはかなり邪魔ですのでこれを短くする方法を記します。
PowerShellのプロンプト文字列は、prompt関数によって定義されていますのでそれの上書きによって変更することができます。

\begin{quote}
\begin{verbatim}

PS C:\Users\E.WAKABAYASHI-AA> function prompt{ "PS>"}

PS>
\end{verbatim}
\end{quote}

プロンプトが「\verb+PS C:\Users\E.WAKABAYASHI-AA>+」から「\verb+PS>+」に変わりました。しかしこのままではPowerShellを立ち上げ直すとまたもとに戻ります。恒久的に変えるにはPowerShell起動時に実行される起動スクリプトにprompt関数を書く必要があります。
起動スクリプトのフルパスは\textbf{\$profile}で取得できます。

\begin{quote}
\begin{verbatim}
PS>$profile
N:\Documents\E.WAKABAYASHI-AA\Documents\WindowsPowerShell\Microsoft.PowerShellISE_profile.ps1
\end{verbatim}
\end{quote}

\subsection{カレントディレクトリの確認}
pwd(Get-Locationコマンドの別名)で確認できます。

\begin{quote}
\begin{verbatim}
PS> pwd

Path                                      
----                                      
C:\Users\E.WAKABAYASHI-AA\Desktop\新しいフォルダー
\end{verbatim}
\end{quote}



\section{ヘルプ}

\subsection{コマンドを探す}

「file」を含むコマンドが表示される。

\begin{quote}
\begin{verbatim}
PS> Get-Command *file*

CommandType     Name                                               Version    Source                       
-----------     ----                                               -------    ------                       
Alias           Write-FileSystemCache                              2.0.0.0    Storage                      
Function        Block-FileShareAccess                              2.0.0.0    Storage                      
Function        Clear-FileStorageTier                              2.0.0.0    Storage                      
Function        Close-SmbOpenFile             
.
.

\end{verbatim}
\end{quote}

\subsection{ヘルプを表示する}

「Get-ChildItem」のヘルプがブラウザ(IE)で表示される。

\begin{quote}
\begin{verbatim}
Get-Help -Online Get-ChildItem
\end{verbatim}
\end{quote}


\section{PowerShellスクリプト}
DOSのバッチ・ファイルに相当する。拡張子は\textbf{.ps1}。権限設定によっては実行ができない。


\subsection{作業の流れ}

\begin{enumerate}
\item 実行権限を\textbf{RemoteSigned}に設定する。
\item 拡張子\textbf{.ps1}を\textbf{Windows PowerShell ISE}に関連づける。
\item 新しいテキスト。ファイルを作成し、拡張子を\textbf{.ps1}に変更する。
\item ダブル・クリックすると\textbf{Windows PowerShell ISE}が開く。

\end{enumerate}

\subsection{デバッガー}
\textbf{Windows PowerShell ISE}には簡単なデバッガーが含まれている。

\hspace{3mm}

\begin{tabular}{ll}
ショートカットキー & メニューと動作内容 \\
\hline
Ctr + 1 & スクリプト ウィンドウを上に表示 \\
Ctr + 2 & スクリプト ウィンドウを右側に表示 \\
Ctr + 3 & スクリプト ウィンドウを最大表示 \\
Ctr + I & スクリプト ウィンドウに移動 \\
Ctr + D & コンソールに移動 \\
F5 & 実行 \\
Shift + F5 & デバッガーを中止 \\
F9 & ブレークポイントの設定/解除 \\
F10 & ステップオーバー \\
\end{tabular}


\section{スクリプトの実行}

スクリプト aaa.ps1 を実行する方法。

\subsection{Windows PowerShell ISE から}
拡張子の関連付けがされていれば aaa.ps1 をダブルクリックでISEが開くのでそのままF5キーを押す。

\subsection{Windows PowerShell から}
ファイル名 aaa.ps1 をコマンドラインにタイプしエンターキーを押す。

\begin{quote}
\begin{verbatim}
PS> aaa.ps1
\end{verbatim}
\end{quote}

\subsection{コマンドプロンプト(DOS窓) から}

powershell.exe -File aaa.ps1とタイプしエンター・キーを押す。

\begin{quote}
\begin{verbatim}
> powershell.exe -File aaa.ps1
\end{verbatim}
\end{quote}


\subsection{コマンドプロンプト(DOS窓) のショートカットから}
ショートカットのプロパティを開き以下の設定を行う。

\begin{itemize}
\item リンク先: \verb+C:\Windows\System32\WindowsPowerShell\v1.0\powershell.exe -NoExit -File aaa.ps1+
\item 作業フォルダー: (空白)
\end{itemize}

powershell.exe をプルパスで指定しているが実際には powershell.exe とだけ入力してOKを押せば自動的にフルパスとなる。

「-NoExit」は実行後もウィンドウを開いたままにしておくオプションである。その他のpowershell.exeのオプションはコマンドプロンプトで「powershell.exe /?」をタイプすると表示される。


\section{パイプとオブジェクト}
PowerShellはパイプでオブジェクトを流すことができます。
この言葉の意味がわからなければPowerShellの真価は分かりません。

\subsection{パイプ}
LINUXのようなUNIX系OSでは、一つの巨大な多機能のプログラムではなく、複数の単機能の小さなプログラムを組み合わせてデータを処理するのが標準的な手法です。
プログラムとプログラムはパイプでつなぎます。先のプログラムの出力がそのまま次のプログラムの入力となるのです。

データの受け渡しはパイプで行われるので中間ファイルを必要としません。文字通りデータがパイプの中を流れるようにプログラムからプログラムへ流れていきます。

UNIXのパイプを流れるのはテキスト(文字)データです。PwerShellのパイプを流れるのはオブジェクトです。より複雑なデータを処理することができます。


\subsection{オブジェクト}
他で多く語られていますのでその定義には触れません。具体的にPowerShellにおけるオブジェクトとは\textbf{.NET Framework}です。ざっくりした表現をすればそれは開発者にとってのWindowsそのものです。一般ユーザーがマウスとキーボードで行う操作をすべてプログラムとして実行することができます。
それ以外にPowerShell独自の\textbf{powershellcustomobject}があります。構造化したテキストデータをパイプに流すことにより柔軟なデータ処理を行うことができます。


\section{関数(Function)}

About Functions \\
\url{https://docs.microsoft.com/en-us/powershell/module/microsoft.powershell.core/about/about_functions?view=powershell-6}

関数は命令文のリストに名前を付けたもの。コマンド・プロンプトから実行できる。
\begin{quote}
A function is a list of PowerShell statements that has a name that you assign. When you run a function, you type the function name. The statements in the list run as if you had typed them at the command prompt.
\end{quote}

\section{Begin, Process, End}

\url{https://docs.microsoft.com/en-us/powershell/module/microsoft.powershell.core/about/about_functions?view=powershell-6#piping-objects-to-functions}

関数はパイプから入力をとることができ、Begin, Process, End のキーワードで動作を制御できる。
\begin{quote}
Piping Objects to Functions
Any function can take input from the pipeline. You can control how a function processes input from the pipeline using Begin, Process, and End keywords. The following sample syntax shows the three keywords:

\begin{verbatim}
function <name> {
  begin {<statement list>}
  process {<statement list>}
  end {<statement list>}
}
\end{verbatim}

\end{quote}

Beginは最初に1回、Endは最後に1回、Processは毎回実行される。

\begin{quote}
The Process statement list runs one time for each object in the pipeline. While the Process block is running, each pipeline object is assigned to the \textbf{\$\_}  automatic variable, one pipeline object at a time.
\end{quote}

\textbf{\$\_} にはパイプライン・オブジェクトが割り当てられる。


\section{オペレーター}
\begin{flushleft}
About Operators \url{https://docs.microsoft.com/en-us/powershell/module/microsoft.powershell.core/about/about_operators?view=powershell-6}
\end{flushleft}

+, -, *, /, \% のような算術オペレータ以外の特殊なオペレータについて記す。

\subsubsection{Call operator \&}

\url{https://docs.microsoft.com/en-us/powershell/module/microsoft.powershell.core/about/about_operators?view=powershell-6#call-operator-}

\begin{quote}
Runs a command, script, or script block. The call operator, also known as the "invocation operator," lets you run commands that are stored in variables and represented by strings or script blocks. The call operator executes in a child scope.

\begin{verbatim}
PS> $c = "get-executionpolicy"
PS> $c
get-executionpolicy
PS> & $c
AllSigned
\end{verbatim}
\end{quote}

コマンド、スクリプト、スクリプト・ブロックを実行する。


\subsubsection{Comma operator ,}

\url{https://docs.microsoft.com/en-us/powershell/module/microsoft.powershell.core/about/about_operators?view=powershell-6#comma-operator-}

\begin{quote}
As a binary operator, the comma creates an array. As a unary operator, the comma creates an array with one member. Place the comma before the member.
\begin{verbatim}
$myArray = 1,2,3
$SingleArray = ,1
\end{verbatim}
\end{quote}

「,」(カンマ)は配列を作成する。
例えば「1,2,3」。「,4」もまた大きさが1の配列である。
カンマの使用で注意すべき点は関数の引数である。
PowerShellの関数の引数は通常のプログラミング言語のように引数間をカンマでくぎらない。
関数というよりはコマンドのように複数の引数は空白で区切る。
もしPowerShellの関数の引数をカンマで区切った場合、
それは複数の引数ではなく、一つの配列であると解釈される。
「command aaa, bbb, ccc」 は引数が3個あるのではなく1個とみなされる。
 


\section{コマンドの使用例}

\subsection{指定したフォルダ内を再帰的に調べる}

子ファイル/ディレクトリを取得するGet-ChildItem コマンドに \textbf{-Recurse}  オプションを付けると取得範囲がサブ・ディレクトリにまで広がります。PowerShellを使う理由の半分はこの\textbf{-Recurse}オプションにあるといっても過言ではありません。

\begin{quote}
\begin{verbatim}
Get-ChildItem -Recurse -LiteralPath <フォルダ>
\end{verbatim}
\end{quote}


\subsection{ディレクトリ情報を取得する。}

\textbf{Get-Item}コマンドでディレクトリ情報を取得します。PowerShellが扱うのはすべてオブジェクトですので取得したディレクトリ情報もまたオブジェクトです。オブジェクトのタイプをしるために\textbf{GetType}コマンドを使用します。
\textbf{System.IO.FileSystemInfo}クラスの詳細はマイクロソフトのサイトで詳しく記されています。
ここではかなり長いパスを指定していますがパス名にもタブ補間が効くので入力は容易です。

\begin{quote}
\begin{verbatim}

PS> Get-Item -LiteralPath \\bsu01119\CureSim\加硫流動シミュレーション\★シミュレーションデータベース\MPD


    ディレクトリ: \\bsu01119\CureSim\加硫流動シミュレーション\★シミュレーションデータベース


Mode                LastWriteTime         Length Name                                                      
----                -------------         ------ ----                                                      
d-----       2019/08/27     13:05                MPD                                                       



PS> (Get-Item -LiteralPath \\bsu01119\CureSim\加硫流動シミュレーション\★シミュレーションデータベース\MPD).GetType()

IsPublic IsSerial Name                                     BaseType                                        
-------- -------- ----                                     --------                                        
True     True     DirectoryInfo                            System.IO.FileSystemInfo                        
\end{verbatim}
\end{quote}


\subsection{エクセルファイルの一覧を取得しCSVファイルに出力する}

行末の 「\textbar」 はパイプです。コマンドをパイプでつなげています。取得したディレクトリ情報からその配下のファイル情報を取得し、拡張子「.xlsx」でフィルタリングし、ファイル名(Name)とフルパス(FullName)と更新日(LastWriteTime)を選び、''filelist.csv''にCSV形式で出力しています。
コマンド・オプションの詳細は「Get-Help -Online \textless コマンド名 \textgreater」で確認できます。

\begin{quote}
\begin{verbatim}
Get-Item -LiteralPath \\bsu01119\CureSim\加硫流動シミュレーション\★シミュレーションデータベース\MPD |
Get-ChildItem -Recurse -Filter *.xlsx |
Select-Object -Property Name, FullName, LastWriteTime |
Export-Csv -NoTypeInformation -Encoding Default -LiteralPath filelist.csv -Force
\end{verbatim}
\end{quote}


\subsection{エクセルファイルの更新日時直近10個を画面で確認する}

前項と同様ディレクトリ情報からファイル情報を取得し、
更新日でソート(\textbf{Sort-Object -Property LastWriteTime -Descending})
した後、表示数を絞り
(\textbf{Select-Object -First 10})
リスト形式(\textbf{Format-List})で表示します。

\begin{quote}
\begin{verbatim}
Get-Item -LiteralPath \\bsu01119\CureSim\加硫流動シミュレーション\★シミュレーションデータベース\MPD |
Get-ChildItem -Recurse -Filter *.xlsx |
Sort-Object -Property LastWriteTime -Descending |
Select-Object -First 10 |
Select-Object -Property Name, FullName, LastWriteTime |
Format-List
\end{verbatim}
\end{quote}


\subsection{CSV形式のファイル名を作成する}

エクセルファイルをCSV形式でエクスポートするとしてそのファイル名を作成します。

ファイル情報すべてに対して処理を行うために\textbf{ForEach-Object}を使います。
\{\}で囲まれたブロックの中では \textbf{\$\_} がパイプから流れてきたオブジェクト(ファイル情報)を表します。
ここで \textbf{pscustomobject} を作成し、プロパティEXCELにはエクセルファイル名、プロパティCSVにはCSV形式のファイル名を格納します。
エクセルファイル名は \textbf{\$\_.Name} です。
CSVファイル名は \textbf{\$\_.Name} の末尾文字列「xlsx」を「csv」に\textbf{-replace}で変換しています。




\begin{quote}
\begin{verbatim}
Get-Item -LiteralPath \\bsu01119\CureSim\加硫流動シミュレーション\★シミュレーションデータベース\MPD |
Get-ChildItem -Recurse -Filter *.xlsx |
Select-Object -First 10 |
ForEach-Object {
    [pscustomobject]@{
        EXCEL = $_.Name
        CSV   = $_.Name -replace "xlsx$", "csv"
    }
}
\end{verbatim}
\end{quote}


\section{EXCELを操作する}

.NET にはCOM相互運用(COM Interop)という仕組みがあります。

\begin{quote}
 .NETには「COM相互運用」と呼ばれる機能があり、COMコンポーネントを手軽に呼び出すことができる。一方、ExcelをはじめとするOffice製品は、その機能をマクロ(VBA)などからも活用できるようにCOMコンポーネントとして実装されている。このため、COM相互運用を使えば.NETアプリケーションから容易にExcelやWordのファイルを開き、それをさまざまに操作することが可能だ。
\vspace{\baselineskip}

\verb+Excelファイルにアクセスするには?[C#、VB]+

\href{https://www.atmarkit.co.jp/ait/articles/0803/06/news147.html}{https://www.atmarkit.co.jp/ait/articles/0803/06/news147.html}

\end{quote}

\vspace{\baselineskip}


この仕組みを利用してPowerShellからEXCELを操作することができます。

\subsection{エクセルファイルを開く閉じる}

エクセルファイルを開いて閉じるだけのスクリプトです。

\begin{quote}
\begin{verbatim}
$filepath = Get-Item -LiteralPath .\bbb.xlsx|Convert-Path
[System.Reflection.Assembly]::LoadWithPartialName("Microsoft.Office.Interop.Excel")|Out-Null
$app = New-Object -ComObject "Excel.Application"
$app.Visible = $true
$book = $app.Workbooks.Open($filepath, $false, $true)
$book.Close($false)
$app.Quit()
\end{verbatim}
\end{quote}

\subsubsection{解説}

\verb+$filepath = Get-Item -LiteralPath .\bbb.xlsx | Convert-Path+

\begin{quote}
ファイル名は「bbb.xlsx」ですがフルパスで指定するので\textbf{Convert-Path}で変換します。
\textbf{Get-Item}で取得したファイル情報をパイプで流しています。
結果(フルパス名)は変数 \verb+$filepath+ に入ります。
\end{quote}

\vspace{\baselineskip}

\verb+[System.Reflection.Assembly]::LoadWithPartialName("Microsoft.Office.Interop.Excel") | Out-Null+

\begin{quote}
相互運用のための準備です。\textbf{Out-Null} でそのとき表示されるメッセージを捨てています。
\end{quote}

\vspace{\baselineskip}

\verb+$app = New-Object -ComObject "Excel.Application"+

\begin{quote}
エクセル・アプリケーション・オブジェクトを取得し変数 \verb+$app+ に格納します。 \\
Application object \\
\href{https://docs.microsoft.com/en-us/office/vba/api/excel.application(object)}{https://docs.microsoft.com/en-us/office/vba/api/excel.application(object)}
\end{quote}


\vspace{\baselineskip}

\verb+$app.Visible = $true+
\begin{quote}
Visible プロパティを真にして表示モードにしています。
\end{quote}


\vspace{\baselineskip}

\verb+$book = $app.Workbooks.Open($filepath, $false, $true)+
\begin{quote}
エクセル・ファイルを読み取り専用で開き、得られたWorkbook オブジェクトを変数 \verb+$book+ に格納します。 \\
Workbooks.Open method \\
\href{https://docs.microsoft.com/en-us/office/vba/api/excel.workbooks.open}{https://docs.microsoft.com/en-us/office/vba/api/excel.workbooks.open}
\end{quote}


\vspace{\baselineskip}

\verb+$book.Close($false)+
\begin{quote}
Workbook オブジェクト(エクセルファイル)を閉じます。
\end{quote}


\vspace{\baselineskip}

\verb+$app.Quit()+
\begin{quote}
エクセルを終了します。
\end{quote}



\subsection{プロセスの確認と終了}
エクセルが起動するとそのプロセスも起動しエクセルの終了と同時にプロセスも終了します。
しかしPowerShellからエクセルを操作するスクリプトを開発していると、デバッガーによる強制終了などが原因でエクセルは終了してもそのプロセスが終了せずにプロセスだけが残ることが度々あります。
そうなるとダブルクリックでエクセルファイルを開いたときそれ以外のファイルも同時に開くという現象が起こります。
そのときはプロセスを終了させる必要があります。


\subsubsection{プロセスの確認}
プロセスは \textbf{Get-Process} で確認します。
エクセルが起動しているときはそのプロセスが表示されます。
もしエクセルが起動していないときにプロセスが表示されればそれは終了しなければなりません。

\begin{quote}
\begin{verbatim}
PS> Get-Process -Name EXCEL

Handles  NPM(K)    PM(K)      WS(K)     CPU(s)     Id  SI ProcessName                                      
-------  ------    -----      -----     ------     --  -- -----------                                      
    794      50    43560      60416       2.94  11964   1 EXCEL 
\end{verbatim}
\end{quote}


\subsubsection{プロセスの終了}

\begin{quote}
\begin{verbatim}
PS> Stop-Process -Name EXCEL
\end{verbatim}
\end{quote}



\subsection{複数ファイルの処理}
単一ファイルの処理ならエクセルVBAを使えばよい。PowerShellは複数ファイルの一括処理ができる。
指定したフォルダ内、-Recurse オプションを使えばサブディレクトリまで含めて一括処理を行うことができる。
フォルダ内のすべてのエクセルファイルのファイル名と1枚目のシート名とそのA1セルの内容を表示するスクリプトを記す。
\vspace{\baselineskip}
\begin{quote}
\begin{verbatim}
Get-ChildItem -LiteralPath .\excelfiles -Filter *.xlsx |
& {
    begin {
        [System.Reflection.Assembly]::LoadWithPartialName("Microsoft.Office.Interop.Excel")|Out-Null
        $app = New-Object -ComObject "Excel.Application"
    }
    process {
        $filepath = $_.FullName|Convert-Path
        $book = $app.Workbooks.Open($filepath, $false, $true)
        $sheet = $book.Worksheets.Item(1)
        [pscustomobject]@{
            FILE  = $_.Name
            SHEET = $sheet.Name
            A1    = $sheet.Range("A1").Text
        }|Write-Output
        $book.Close($false)
    }
    end {
        $app.Quit()
    }
}
\end{verbatim}
\end{quote}

Get-ChildItem で取得したエクセルファイル情報をパイプでスクリプトブロックに流す。
スクリプト・ブロックにはコール・オペレータ「\&」を付けて実行可能にしている。
begin ブロックでエクセル・オブジェクトを取得し、processブロックで各ファイルに対する処理を行い、最後にendブロックでエクセル・オブジェクトを終了する。








\section{使用例(スクリプト)}

%--------------------------------------------------------------------
\subsection{ファイルの一覧を取得する}

\begin{verbatim}
$modelcsv = Get-Item -LiteralPath work|Convert-Path|Join-Path -ChildPath "model.csv"

$d1 = Get-Item -LiteralPath "\\bsu01119\CureSim\加硫流動シミュレーション\★シミュレーションデータベース\MPD" $d2 = Get-Item -LiteralPath "\\bsu01119\CureSim\加硫流動シミュレーション\★シミュレーションデータベース\proto"

$d1, $d2|
ForEach-Object{
    $_|Get-ChildItem -Recurse -File -Filter d3plot
}|
ForEach-Object {
    "."|Write-Host -NoNewline

    [pscustomobject]@{
        MODEL  = $_.FullName.Split('\')[7] -replace " .+$", ""
        D3PLOT = $_.FullName
        DIR    = $_|Convert-Path|Split-Path -Parent
    }
}|
ForEach-Object {
    foreach ($i in Get-ChildItem -LiteralPath $_.DIR -Filter *.xls?) {
        [pscustomobject]@{
            MODEL  = $_.MODEL
            DIR    = $_.DIR
            D3PLOT = $_.D3PLOT
            EXCEL  = $i.FullName
        }|Write-Output
    }
}|
#Format-List
Export-Csv -Encoding Default -NoTypeInformation -LiteralPath work\model.csv

""|Write-Host

\end{verbatim}



%--------------------------------------------------------------------
\subsection{ファイル・リスト内のエクセル・ファイルのシート名を取得する}

\begin{verbatim}
Import-Csv -Encoding Default -LiteralPath .\work\model.csv|
Where-Object {($_.EXCEL|Split-Path -Leaf) -cmatch "_pressure_PLY"}|
&{
    begin {
        [System.Reflection.Assembly]::LoadWithPartialName("Microsoft.Office.Interop.Excel")|Out-Null
        $excel = New-Object -ComObject "Excel.Application"
        $excel.Visible = $false
    }

    process {
        "."|Write-Host -NoNewline
        $o = $_
        try {
            $book  = $excel.Workbooks.Open($o.EXCEL, $false, $true)
            $book.Sheets|
            ForEach-Object {
                [pscustomobject]@{
                    MODEL  = $o.MODEL
                    SHEET  = $_.Name
                    EXCEL  = $o.EXCEL
                    D3PLOT = $o.D3PLOT
                    DIR    = $o.DIR
                }|Write-Output
            }
            $book.Close($false)
        } catch {
            "Error"
            $o.EXCEL
            exit 1
        }        
    }

    end {
        $excel.Quit()        
    }
}| 
#Format-List
Export-Csv -NoTypeInformation -Encoding Default -LiteralPath "work\sheet_ply.csv"

""|Write-Host
\end{verbatim}



%--------------------------------------------------------------------
\subsection{エクセル・シートの指定した領域をCSVファイルに書き出す}

\begin{verbatim}

$sim = Get-Item -LiteralPath .\work\sim1.xlsx

# セル内改行を取り除く
function remove_crlf_comma($str) {
    $a = ""
    ([string]$str).GetEnumerator().ForEach({
        # 制御コードを除外
        if ([System.Char]::GetUnicodeCategory($_) -ne [System.Globalization.UnicodeCategory]::Control) {
            if ($_ -eq ',') {
                $a += '、'
            } else {
                $a += $_
            }
        }
    })
    return $a
}


[System.Reflection.Assembly]::LoadWithPartialName("Microsoft.Office.Interop.Excel")|Out-Null
$excel = New-Object -ComObject "Excel.Application"
$excel.Visible = $false

$book  = $excel.Workbooks.Open($sim.FullName, $false, $true)
$excel.ActiveWindow.FreezePanes = $false

$sheet = $book.Sheets("全体管理表")

$model_rg = $sheet.Range("B8")
$jul1_rg  = $sheet.Range("Z8")
$dec31_rg  = $sheet.Range("HF8")
$tanto_rg  = $sheet.Range("T8")
$status_rg = $sheet.Range("U8")

if ($model_rg.Text -ne "モデルNo.") {
    "セルが変更された。"
    exit 1
}

if ($jul1_rg.Text -ne "1/1") {
    "セルが変更された。"
    exit 1
}

if ($dec31_rg.Text -ne "7/8") {
    "セルが変更された。"
    exit 1
}

if ($tanto_rg.Text -ne "DR担当") {
    "セルが変更された。"
    exit 1
}

if ($status_rg.Text -ne "進捗") {
    "セルが変更された。"
    exit 1
}



$header_row = $model_rg.Row
$top_row    = $header_row + 1  
$bottom_row = $model_rg.End([Microsoft.Office.Interop.Excel.XlDirection]::xlDown).Row

$model_col = $model_rg.Column

$date_rgs = $sheet.Range($jul1_rg, $dec31_rg)

$jul1_col  = $jul1_rg.Column
$dec31_col = $dec31_rg.Column


$r1 = $sheet.Cells($header_row, $jul1_col)
$r2 = $sheet.Cells($header_row, $dec31_col)
$DATELINE = $sheet.Range($r1, $r2)

function find($str, $range_of_line) {

    # "0"とかなら数値型に変換する。
    if ($str -match "^\d+$"){
        $str = [int]$str
    }

    try{
        $match     = $excel.WorksheetFunction.Match($str, $range_of_line, 0)
        $match_rg  = $excel.WorksheetFunction.Index($DATELINE, $match)
        $match_day = $excel.WorksheetFunction.Text($match_rg, "yyyy-mm-dd")
    }
    catch{
        $match_day = ""
    }
    return $match_day
}


$top_row .. $bottom_row|
ForEach-Object {
    $r = $_
    $r1 = $sheet.Cells($r, $jul1_col)
    $r2 = $sheet.Cells($r, $dec31_col)
    $rg12 = $sheet.Range($r1, $r2)

        $model    = $sheet.Cells($r, $model_col).Text
        $size     = $sheet.Cells($r, $model_col + 1).Text
        $seisan   = $sheet.Cells($r, $model_col + 4).Text
        $process  = $sheet.Cells($r, $model_col + 5).Text
        $factory  = $sheet.Cells($r, $model_col + 6).Text
        $mokuteki = $sheet.Cells($r, $model_col + 10).Text
        $tanto    = $sheet.Cells($r, $tanto_rg.Column).Text
        $status   = $sheet.Cells($r, $status_rg.Column).Text
        $zeroday  = find "0"  $rg12
        $drday    = find "DR" $rg12

        $model    = remove_crlf_comma $model
        $size     = remove_crlf_comma $size
        $seisan   = remove_crlf_comma $seisan
        $process  = remove_crlf_comma $process
        $factory  = remove_crlf_comma $factory
        $mokuteki = remove_crlf_comma $mokuteki
        $tanto    = remove_crlf_comma $tanto
        $status   = remove_crlf_comma $status
        $zeroday  = remove_crlf_comma $zeroday
        $drday    = remove_crlf_comma $drday


    [pscustomobject]@{
		MODEL    = $model
		SIZE     = $size
		SEISAN   = $seisan
		PROCESS  = $process
		FACTORY  = $factory
		MOKUTEKI = $mokuteki
		TANTO    = $tanto
		STATUS   = $status
		ZERODAY  = $zeroday
		DRDAY    = $drday
    }
}|Export-Csv -Encoding Default -NoTypeInformation -LiteralPath .\work\sim1.csv

$book.Close($false)
$excel.Quit()        

\end{verbatim}

%-----------------------------------------------------------------
\subsection{ダイアログを表示しパラメータをYAMLファイルに保存する。}

\begin{verbatim}
$version = "BareY2_ver4.9.0"  # 複数.dyn出力をリスト化
#$version = "BareY2_ver4.8.0"  # ノード並びチェック追加
#$version = "BareY2_ver4.7.0"  # VH位置(隣ノード)を最終タイムステートで検出
#$version = "BareY2_ver4.6.1"  # ノード並びテーブルからXYZ座標値削除
#$version = "BareY2_ver4.6.0"  # 環境変数でなくlog.yamlを介してパラメータを渡す
#$env:VERSION = "BareY2_ver4.5.1"  # LS-PREPOT, RScriptはサーバー上を利用
#$env:VERSION = "BareY2_ver4.4.1"  # debug_disp_node.Rmd 他
#$env:VERSION = "BareY2_ver4.4.0"  # モールド・エレメントの重複に対応
#$env:VERSION = "BareY2_ver4.3.1"  # areavh.R STATE_NOの重複削除忘れ
#$env:VERSION = "BareY2_ver4.3.0"  # work=>resultsを -Recurse で
#$env:VERSION = "BareY2_ver4.2.0"  # モールド・ノード始点検出方法修正
#$env:VERSION = "BareY2_ver4.1.0"  # '$'を含むサーバ・パスに対応
#$env:VERSION = "BareY2_ver4.0.0"  # レンジ毎の最遠タイヤ・ノード
#$env:VERSION = "BareY2_ver3.4.0"  # VHLX, VHRY 自動設定
#$env:VERSION = "BareY2_ver3.3.0"  # 測定範囲指定変更(VHL=>VHLX, VHR=>VHRY)
#$env:VERSION = "BareY2_ver3.2.0"  # 測定範囲をユーザが指定(VHL, VHR)
#$env:VERSION = "BareY2_ver3.1.0"  # 最終ステートの面積も表示
#$env:VERSION = "BareY2_ver3.0.0"  # 辺で端ノード抽出
#$env:VERSION = "BareY2_ver2.0.0"  # 面積
#$env:VERSION = "BareY2_ver1.2.1"  # デバッグ用に測定ノード(intfor_nodelist)を表示
#$env:VERSION = "BareY2_ver1.2.0"  # 最終STATE_NOを指定する。
#$env:VERSION = "BareY2_ver1.1.2"  # area.htmlをnmake.exeからではなくSTART.ps1から開く。
#$env:VERSION = "BareY2_ver1.1.1"  # MakefileのUsage変更

if ($args[0] -eq "-v") {
    $env:VERSION
    exit
}

$simdir = Get-Item -LiteralPath \\bsu01119\CureSim\加硫流動シミュレーション\★シミュレーションデータベース
#$simdir = Get-Item -LiteralPath "\\Bcf19065\85\N4600\001_共通管理\006_4650\102_登録テーマ別資料\18-4652-PD-P-05_不良予測技術開発\(2)開発諸元(G1)\〇Simベア間エア予測\構造&成型機種別_予測精査\サイドベア予測プログラム_評価\20190313_MTG\2.検証サイズ選定"
#$simdir = Get-Item -LiteralPath "\\BCF19064\N4600a`$\001_共通管理\006_4650\102_登録テーマ別資料\18-4652-PD-P-05_不良予測技術開発\(2)開発諸元(G1)\〇Simベア間エア予測\構造&成型機種別_予測精査\サイドベア予測プログラム_評価\20190313_MTG\2.検証サイズ選定"

[void][System.Reflection.Assembly]::LoadWithPartialName("System.Windows.Forms")

$form1      = New-Object -TypeName System.Windows.Forms.Form
$btn_d3plot = New-Object -TypeName System.Windows.Forms.Button
$btn_vhline = New-Object -TypeName System.Windows.Forms.Button
$btn_ok     = New-Object -TypeName System.Windows.Forms.Button
$lbl_d3plot = New-Object -TypeName System.Windows.Forms.Label
$lbl_intfor = New-Object -TypeName System.Windows.Forms.Label
$lbl_vhline = New-Object -TypeName System.Windows.Forms.Label
$lbl_last   = New-Object -TypeName System.Windows.Forms.Label
$lbl_vhl    = New-Object -TypeName System.Windows.Forms.Label
$lbl_vhr    = New-Object -TypeName System.Windows.Forms.Label
$tbox_last  = New-Object -TypeName System.Windows.Forms.TextBox
$tbox_vhl   = New-Object -TypeName System.Windows.Forms.TextBox
$tbox_vhr   = New-Object -TypeName System.Windows.Forms.TextBox
$ofd_d3plot = New-Object -TypeName System.Windows.Forms.OpenFileDialog
$ofd_vhline = New-Object -TypeName System.Windows.Forms.OpenFileDialog

$font = [System.Drawing.Font]::new($lbl_d3plot.Font.FontFamily.Name, 11)
$btn_d3plot.Font = $font
$btn_vhline.Font = $font
$btn_ok.Font     = $font
$lbl_d3plot.Font = $font
$lbl_intfor.Font = $font
$lbl_vhline.Font = $font
$lbl_last.Font   = $font
$lbl_vhl.Font    = $font
$lbl_vhr.Font    = $font

$form1.Text   = $env:VERSION
$form1.Width  = 850
$form1.Height = 550

$btn_d3plot.Text = "d3plot"
$btn_vhline.Text = "VH_line.cfile"
$btn_ok.Text     = "OK"
$lbl_d3plot.Text = "D3PLOT = "
$lbl_intfor.Text = "INTFOR = "
$lbl_vhline.Text = "VHLINE = "
$lbl_last.Text   = "LAST_STATE_NO(最終タイム・ステート):"
$lbl_vhl.Text    = "VHLX(測定左端X値[mm])[空白なら自動設定]:"
$lbl_vhr.Text    = "VHRY(測定右下端Y値[mm][空白なら自動設定]):"

$btn_d3plot.Top    = 0
$lbl_d3plot.Top    = 40
$lbl_intfor.Top    = 110
$btn_vhline.Top    = 180
$lbl_vhline.Top    = 210
$lbl_last.Top      = 280
$tbox_last.Top     = 280
$lbl_vhl.Top       = 340
$tbox_vhl.Top      = 340
$lbl_vhr.Top       = 380
$tbox_vhr.Top      = 380
$btn_ok.Top        = 430

$tbox_last.Left    = 250
$tbox_vhl.Left    = 350
$tbox_vhr.Left    = 350

$lbl_d3plot.Height = 50
$lbl_intfor.Height = 50
$lbl_vhline.Height = 50

$btn_d3plot.Width  = 120
$btn_vhline.Width  = 120
$btn_ok.Width      = 120

$lbl_d3plot.Width  = 800
$lbl_intfor.Width  = 800
$lbl_vhline.Width  = 800
$lbl_last.Width    = 250
$lbl_vhl.Width    = 350
$lbl_vhr.Width    = 350


$ofd_d3plot.InitialDirectory = $simdir.FullName
$ofd_d3plot.FileName = "d3plot"

$tbox_last.Text = "1550"


$btn_d3plot_click = {
    if ($ofd_d3plot.ShowDialog() -eq [System.Windows.Forms.DialogResult]::OK) {
        $script:d3plot = $ofd_d3plot.FileName
        if (Test-Path -LiteralPath $script:d3plot) {
            $script:intfor = [System.IO.Path]::GetDirectoryName($script:d3plot) + "\intfor"
            if (Test-Path -LiteralPath $script:intfor) {
                $lbl_d3plot.Text += $script:d3plot
                $lbl_intfor.Text += $script:intfor
            }
        }
    }
}
$btn_d3plot.Add_Click($btn_d3plot_click)

$btn_vhline_click = {
    if ($ofd_vhline.ShowDialog() -eq [System.Windows.Forms.DialogResult]::OK) {
        $script:vhline = $ofd_vhline.FileName
        if (Test-Path -LiteralPath $script:vhline) {
            $lbl_vhline.Text += $script:vhline
        }
    }
}
$btn_vhline.Add_Click($btn_vhline_click)

$btn_ok_click = {
    $form1.Close()
}
$btn_ok.Add_Click($btn_ok_click)


$form1.Controls.Add($btn_d3plot)
$form1.Controls.Add($lbl_d3plot)
$form1.Controls.Add($lbl_intfor)
$form1.Controls.Add($btn_vhline)
$form1.Controls.Add($lbl_vhline)
$form1.Controls.Add($lbl_last)
$form1.Controls.Add($tbox_last)
$form1.Controls.Add($lbl_vhl)
$form1.Controls.Add($tbox_vhl)
$form1.Controls.Add($lbl_vhr)
$form1.Controls.Add($tbox_vhr)
$form1.Controls.Add($btn_ok)

$form1.ShowDialog() |Out-Null


[pscustomobject]@{
    VERSION       = $version
    DATE          = [string](Get-Date -Format g)
    D3PLOT        = $script:d3plot
    INTFOR        = $script:intfor
    VHLINE        = $script:vhline
    LAST_STATE_NO = $tbox_last.Text
    VHLX          = $tbox_vhl.Text
    VHRY          = $tbox_vhr.Text
}|
ConvertTo-Yaml|
Set-Content -LiteralPath "work\log.yaml"


\end{verbatim}


%-----------------------------------------------------------------
\subsection{CSVファイルを連結する}

\begin{verbatim}
$sim1 = Get-Content -Encoding Default -LiteralPath .\work\sim1.csv

$sim7 = Get-Content -Encoding Default -LiteralPath .\work\sim7.csv|Select-Object -Skip 1

$sim1 + $sim7|Set-Content -Encoding Default -LiteralPath work/sim.csv

\end{verbatim}




\end{document}
